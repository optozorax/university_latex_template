\PassOptionsToPackage{usenames,dvipsnames}{xcolor}

\documentclass{universityreport}

% Стили из этого репозитория
\usepackage{fixint}
\usepackage{rumathgrk}
\usepackage{labformula}
\usepackage[pretty]{labcode}

% Выставляем мета-информацию
\hypersetup{
    pdfauthor={Шепрут И.И.},
    pdftitle={Шаблон для ВКР}
}

\usepackage{pdfpages}
\usepackage{graphicx}

\begin{document}

\includepdf[pages={1}]{diploma_title.pdf}

\tableofcontents\fixTexttTextscSpaceBug

\addchap{Введение}

В данном документе показывается как можно использовать данный стиль для написания отчётов или диплома в \LaTeX.

\chapter{Аннотация}

Основные фичи данного документа:

\begin{itemize}
	\item Возможность выбора стиля оформления ГОСТ/не ГОСТ.
	\item Красивый стиль для не ГОСТ компиляции.
	\item Специальное окружение для вставки картинок, таблиц и формул по ГОСТУ.
	\item Возможность компиляции через \texttt{XeLaTeX} и \texttt{pdfLaTeX}.
\end{itemize}

\chapter{Главища}

\section{Разные шрифты}

На следующих строках показано использование разных шрифтов.

\texttt{textrm}: \textrm{The quick brown fox jumps over the lazy dog. Съешь ещё этих мягких французских булок, да выпей же чаю.}

\texttt{textit}: \textit{The quick brown fox jumps over the lazy dog. Съешь ещё этих мягких французских булок, да выпей же чаю.} 

\texttt{textbf}: \textbf{The quick brown fox jumps over the lazy dog. Съешь ещё этих мягких французских булок, да выпей же чаю.}

\texttt{textbf+textit}: \textbf{\textit{The quick brown fox jumps over the lazy dog. Съешь ещё этих мягких французских булок, да выпей же чаю.}}

\texttt{textsl}: \textsl{The quick brown fox jumps over the lazy dog. Съешь ещё этих мягких французских булок, да выпей же чаю.}

\texttt{textsc}: \textsc{The quick brown fox jumps over the lazy dog. Съешь ещё этих мягких французских булок, да выпей же чаю.}

\texttt{texttt}: \texttt{The quick brown fox jumps over the lazy dog. Съешь ещё этих мягких французских булок, да выпей же чаю.}

\texttt{textsf}: \textsf{The quick brown fox jumps over the lazy dog. Съешь ещё этих мягких французских булок, да выпей же чаю.}

\section{Использование рисунка}

Мику Хацунэ, или Хацунэ Мику --- японская виртуальная певица, созданная компанией Crypton Future Media 31 августа 2007 года. Для синтеза её голоса используется технология семплирования голоса живой певицы с использованием программы Vocaloid компании Yamaha Corporation. Голосовым провайдером послужила японская сейю Саки Фудзита. Оригинальный образ был создан японским иллюстратором KEI Garou, также работавшим над внешностью других вокалоидов для Crypton Future Media. Диски с песнями Мику завоевывали первые позиции в японских чартах. Она является самым известным и популярным вокалоидом и стала поп-идолом. Также, благодаря технологии псевдообъемной проекции на полупрозрачный экран, она даёт и живые концерты. На её страницу в Facebook подписано свыше 2,5 миллиона пользователей. Изображена на \ref{miku}.

\begin{gostimage}{Hatsune Miku}{miku}
	\includegraphics[width=0.5\textwidth]{miku.png}
\end{gostimage}

Мику Хацунэ 16 лет, при росте 1м 58см она весит 42кг. Цвет волос аквамариновый. Мику иногда изображают с луком-пореем, который стал её отличительным атрибутом. Мику используется во многих произведениях искусства, рекламе, играх и так далее.

\section{Использование таблицы}

Электронная таблица --- компьютерная программа, позволяющая проводить вычисления с данными, представленными в виде двумерных массивов, имитирующих бумажные таблицы. Некоторые программы организуют данные в <<листы>>, предлагая, таким образом, третье измерение. Пример таблицы \ref{bigtable}.

\begin{gosttable}{A simple table}{bigtable}
	\begin{tabular}{| c c c c c c c c c c c c c c c c |}
		\hline
		1 & 2 & 3 & 1 & 2 & 3 & 1 & 2 & 3 & 1 & 2 & 3 & 1 & 2 & 3 & 1 \\
		4 & 5 & 6 & 1 & 2 & 3 & 1 & 2 & 3 & 1 & 2 & 3 & 1 & 2 & 3 & 6 \\
		7 & 8 & 9 & 1 & 2 & 3 & 1 & 2 & 3 & 1 & 2 & 3 & 1 & 2 & 3 & 5 \\
		\hline
	\end{tabular}
\end{gosttable}

Электронные таблицы (ЭТ) представляют собой удобный инструмент для автоматизации вычислений. Многие расчёты, в частности в области бухгалтерского учёта, выполняются в табличной форме: балансы, расчётные ведомости, сметы расходов и т. п. 

\section{Использование формул}

Математическая формула (от лат. formula — уменьшительное от forma — образ, вид) — в математике, а также физике и прикладных науках, символическая запись высказывания (которое выражает логическое суждение), либо формы высказывания. Пример формулы смотри в \eqref{ololo}.

\begin{equation}\label{ololo}
\cubr{1 + \sqbr{2 + \roubr{3 + 4}}} = 10
\end{equation}

Формула, наряду с термами, является разновидностью выражения формализованного языка. В более широком смысле формула — всякая чисто символьная запись (см. ниже), противопоставляемая в математике различным выразительным способам, имеющим геометрическую коннотацию: чертежам, графикам, диаграммам, графам и т. п.

\conclusion \eqref{ololo} равно 10

Специальное обозначение \texttt{vc} для векторов:

$$
\vc{x} + \vc{y} = \vc{z}
$$

Так же есть русские обозначения для следующих функций:

$$
\sin \cos \tg \ctg \arcsin \sh \rugrad \rurot \rudiv \lim \ln
$$

\subsection{fixint}

Использование пакета \texttt{fixint} для того, чтобы знак интеграла был прямым:

$$
\ruintegral{a}{b} \sin x \rud{x} = \rudiv \rugrad \vc{y} = \fpartial{x}{t} \cdot \fpartialpow{y}{t}{3}
$$

\section{Использование вставки кода}

Код — взаимно однозначное отображение конечного упорядоченного множества символов, принадлежащих некоторому конечному алфавиту, на иное, не обязательно упорядоченное, как правило более обширное множество символов для кодирования передачи, хранения или преобразования информации.

\codeinput{C++}{vector2.h}{vector2.h}

Например, код Морзе, в котором любая буква/символ кодируются последовательностями точек и тире. Иной пример — кодирование букв, чисел и символов последовательностями логических нулей и единиц в компьютерах. Последовательность элементарных закодированных символов принято называть кодовым сообщением или кодовой посылкой. Иногда последовательность закодированных символов известной длины называют кодовым словом, или кодовым кадром. 

\section{Списки всякой фигни}

\listoffigures
\listoftables

\addcontentsline{toc}{chapter}{Список использованных источников}
% Переименовываем "Литература" в "Список использованных источников"
\renewcommand\bibname{Список использованных источников}
\begin{thebibliography}{00}

	\bibitem{fem}
		Соловейчик, Ю.Г. Метод конечных элементов для скалярных и векторных задач /
		Ю.Г. Соловейчик, М.Э. Рояк, М.Г. Персова. -- Новосибирск : НГТУ, 2007. -- 869 с.
		
	\bibitem{fem_zienkiewicz}
		Zienkiewicz, O.C. The Finite Element Method: Its Basis and Fundamentals (Sixth Edition) / 
		O.C. Zienkiewicz, R.L. Taylor, J.Z. Zhu -- Butterworth-Heinemann, 2005. -- 250 pp.
		
\end{thebibliography}

\pagebreak

\appendix

\chapter{Что такое четвёртое измерение?}

4D, четырёхмерное пространство, четвёртое измерение - абстрактная концепция пространства, обобщающего свойства 3D, 2D, 1D, 0D на более высокую размерность.

Мы рассуждаем в первую очередь о четырёхмерном пространстве как о чём-то, максимально похожим на наше трёхмерное, но имеющее размерность 4. Не важно каким образом оно может реализовываться в математике или в программировании, главное чтобы сохранялась идейная похожесть на трёхмерный мир.

Как математическая модель четырёхмерного пространства для расчётов и симуляций отлично подходит евклидово пространство и аналитическая геометрия.

\chapter{Часто задаваемый вопрос}

\textbf{А что насчёт времени? Время - это же четвёртая ось, да? Так Эйнштейн говорил.}

Не надо путать тёплое с мягким. Четырёхмерное пространство - это в первую очередь абстрактная концепция, которая существует на бумаге, в программах и умах людей. 4D никак не зависит от нашего мира.

А время как четвёртая ось - одно из практических применений абстрактного четырёхмерного пространства в физике. Но там используется не чистое 4D, а пространство Минковского.

Так же здесь не будут рассматриваться никакие гипотезы о том, что в нашем мире четвёртое измерение позволяет хранить рай и ад или параллельные миры, это абсолютно неинтересно, недоказуемо и идите на рентв со своим бредом.

\end{document}