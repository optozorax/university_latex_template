\documentclass{diploma}
\usepackage{nstu}

\hypersetup{
    pdfauthor={Шепрут И.И.},
    pdftitle={Шаблон для ВКР}
}

\usepackage{graphicx}

\begin{document}

\tableofcontents\fixTexttTextscSpaceBug

\addchap{Введение}

The quick brown fox jumps over the lazy dog. Съешь ещё этих мягких французских булок, да выпей же чаю.The quick brown fox jumps over the lazy dog. Съешь ещё этих мягких французских булок, да выпей же чаю.The quick brown fox jumps over the lazy dog. Съешь ещё этих мягких французских булок, да выпей же чаю.The quick brown fox jumps over the lazy dog. Съешь ещё этих мягких французских булок, да выпей же чаю.The quick brown fox jumps over the lazy dog. Съешь ещё этих мягких французских булок, да выпей же чаюкмвакмвакмвамквакмвакмвакмвакмвакмва.
aoeuaoeu

\chapter{Аннотация}

\textrm{The quick brown fox jumps over the lazy dog. Съешь ещё этих мягких французских булок, да выпей же чаю.The quick brown fox jumps over the lazy dog. Съешь ещё этих мягких французских булок, да выпей же чаю.The quick brown fox jumps over the lazy dog. Съешь ещё этих мягких французских булок, да выпей же чаю.The quick brown fox jumps over the lazy dog. Съешь ещё этих мягких французских булок, да выпей же чаю.The quick brown fox jumps over the lazy dog. Съешь ещё этих мягких французских булок, да выпей же чаюкмвакмвакмвамквакмвакмвакмвакмвакмва.}

\textit{The quick brown fox jumps over the lazy dog. Съешь ещё этих мягких французских булок, да выпей же чаю.The quick brown fox jumps over the lazy dog. Съешь ещё этих мягких французских булок, да выпей же чаю.The quick brown fox jumps over the lazy dog. Съешь ещё этих мягких французских булок, да выпей же чаю.The quick brown fox jumps over the lazy dog. Съешь ещё этих мягких французских булок, да выпей же чаю. Смотри рисунок \ref{kladenets}.} 

\begin{gostimage}{Клавиатура Кладенец}{kladenets}
	\includegraphics[width=0.5\textwidth]{kladenets.jpg}
\end{gostimage}

\textbf{The quick brown fox jumps over the lazy dog. Съешь ещё этих мягких французских булок, да выпей же чаю.The quick brown fox jumps over the lazy dog. Съешь ещё этих мягких французских булок, да выпей же чаю.The quick brown fox jumps over the lazy dog. Съешь ещё этих мягких французских булок, да выпей же чаю.The quick brown fox jumps over the lazy dog. Съешь ещё этих мягких французских булок, да выпей же чаю. Смотри таблицу \ref{bigtable}.}

\begin{gosttable}{A simple table}{bigtable}
	\begin{tabular}{| c c c c c c c c c c c c c c c c |}
		\hline
		1 & 2 & 3 & 1 & 2 & 3 & 1 & 2 & 3 & 1 & 2 & 3 & 1 & 2 & 3 & 1 \\
		4 & 5 & 6 & 1 & 2 & 3 & 1 & 2 & 3 & 1 & 2 & 3 & 1 & 2 & 3 & 6 \\
		7 & 8 & 9 & 1 & 2 & 3 & 1 & 2 & 3 & 1 & 2 & 3 & 1 & 2 & 3 & 5 \\
		\hline
	\end{tabular}
\end{gosttable}

\textbf{\textit{The quick brown fox jumps over the lazy dog. Съешь ещё этих мягких французских булок, да выпей же чаю.The quick brown fox jumps over the lazy dog. Съешь ещё этих мягких французских булок, да выпей же чаю.The quick brown fox jumps over the lazy dog. Съешь ещё этих мягких французских булок, да выпей же чаю.The quick brown fox jumps over the lazy dog. Съешь ещё этих мягких французских булок, да выпей же чаю.}}

\textsl{The quick brown fox jumps over the lazy dog. Съешь ещё этих мягких французских булок, да выпей же чаю.The quick brown fox jumps over the lazy dog. Съешь ещё этих мягких французских булок, да выпей же чаю.The quick brown fox jumps over the lazy dog. Съешь ещё этих мягких французских булок, да выпей же чаю.The quick brown fox jumps over the lazy dog. Съешь ещё этих мягких французских булок, да выпей же чаю.The quick brown fox jumps over the lazy dog. Съешь ещё этих мягких французских булок, да выпей же чаю.}

\textsc{The quick brown fox jumps over the lazy dog. Съешь ещё этих мягких французских булок, да выпей же чаю.The quick brown fox jumps over the lazy dog. Съешь ещё этих мягких французских булок, да выпей же чаю.The quick brown fox jumps over the lazy dog. Съешь ещё этих мягких французских булок, да выпей же чаю.The quick brown fox jumps over the lazy dog. Съешь ещё этих мягких французских булок, да выпей же чаю.}

\texttt{The quick brown fox jumps over the lazy dog. Съешь ещё этих мягких французских булок, да выпей же чаю.The quick brown fox jumps over the lazy dog. Съешь ещё этих мягких французских булок, да выпей же чаю.The quick brown fox jumps over the lazy dog. Съешь ещё этих мягких французских булок, да выпей же чаю.The quick brown fox jumps over the lazy dog. Съешь ещё этих мягких французских булок, да выпей же чаю.}

\textsf{The quick brown fox jumps over the lazy dog. Съешь ещё этих мягких французских булок, да выпей же чаю.The quick brown fox jumps over the lazy dog. Съешь ещё этих мягких французских булок, да выпей же чаю.The quick brown fox jumps over the lazy dog. Съешь ещё этих мягких французских булок, да выпей же чаю.The quick brown fox jumps over the lazy dog. Съешь ещё этих мягких французских булок, да выпей же чаю.The quick brown fox jumps over the lazy dog. Съешь ещё этих мягких французских булок, да выпей же чаю.}

\section{Под-аннотация}

\begin{equation}\label{ololo}
\cubr{1 + \sqbr{2 + \roubr{3 + 4}}} = 10
\end{equation}

\conclusion \eqref{ololo} равно 10

$$
\vc{x} + \vc{y} = \vc{z}
$$

$$
\ruintegral{a}{b} \sin x \rud{x} = \rudiv \rugrad \vc{y} = \fpartial{x}{t} \cdot \fpartialpow{y}{t}{3}
$$

\subsection{Под-под-аннотация}

кмва
кмва
кмва

\textbf{example.cpp}\\
\lstinputlisting[language=C++,style=listing_C,inputencoding=cp1251]{example.cpp}

\chapter{Что-то}

\begin{equation}\label{ololo2}
\cubr{1 + \sqbr{2 + \roubr{3 + 4}}} = \norm{10}
\end{equation}

\conclusion \ref{ololo2} равно 10

aoesunthaoesunthaoeunsoaeuhteoausntohustnaoh
u ntsoa uhsntao uhNTSsn taoeush aoenhaoesnt hoaetns oesatnhsnthoe usnth oaeusthoea usntoehu snoetu oaensh eoasnhoeausteoahuntsohu
\begin{equation}
\sin \cos \tg \ctg \arcsin \sh \rugrad \rurot \rudiv \lim \ln
\end{equation}
aoesunthaoesunthaoeunsoaeuhteoausntohustnaoh
u ntsoa uhsntao uhNTSsn taoeush aoenhaoesnt hoaetns oesatnhsnthoe usnth oaeusthoea usntoehu snoetu oaensh eoasnhoeausteoahuntsohu

\begin{gostimage}{Hatsune Miku}{miku}
	\includegraphics[width=0.5\textwidth]{miku.png}
\end{gostimage}

\begin{gosttable}{A simple table}{simple}
	\begin{tabular}{| l c r |}
		\hline
		1 & 2 & 3 \\
		4 & 5 & 6 \\
		7 & 8 & 9 \\
		\hline
	\end{tabular}
\end{gosttable}

\section{Списки всякой фигни}

\listoffigures
\listoftables

\end{document}