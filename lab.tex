\PassOptionsToPackage{usenames,dvipsnames}{xcolor}

\documentclass{universityreport}

% Стили из этого репозитория
\usepackage{fixint}
\usepackage{rumathgrk}
\usepackage{labformula}
\usepackage{labtitle}
\usepackage[pretty]{labcode}

% Выставляем мета-информацию
\hypersetup{
    pdfauthor={Шепрут И.И.},
    pdftitle={Шаблон для Лабораторной работы}
}

\usepackage{pdfpages}
\usepackage{graphicx}

\begin{document}

% Создаём титульный лист лабораторной работы
\labtitlepage{%
% Название кафедры
прикладной математики
}{%
Лабораторная работа №1
}{%
по дисциплине <<Создание идеальных отчётов в \LaTeX>>
}{% Название лабораторной
Отчёт божественного качества
}{% Название факультета
ПМИ
}{% Номер группы
ПМ-63
}{% Имя студента
Шепрут И.И.
}{% Номер варианта
42
}{% Имя преподавателя
Шепрут И.И.
}

\chapter{Введение}\fixTexttTextscSpaceBug

Весь основной функционал показан в файле диплома. Здесь же показан дополнительный функционал для лабораторных работ. В нём используется титульник из пакета \codeinline{TeX}{\usepackage{labtitle}} при помощи следующего кода:

\codeinput{TeX}{title_example.tex}{title\_example.tex}

Так же можно вставлять код прямо в тексте наподобие \codeinline{C++}{int a = 0;} при помощи \codeinline{TeX}{\codeinline{C++}{int a = 0;}}. 

Здесь, в отличие от файла диплома используется \codeinline{TeX}{\usepackage[pretty]{labcode}}, когда там просто \codeinline{TeX}{\usepackage{labcode}}. Предпочтительней использовать первый вариант, потому что он мощнее по возможностям и красивей выглядит.

Для кода из файлов используется минимально разрешённый размер шрифта --- 8pt.

Пример вставки кода из файла:

\codeinput{C++}{vector2.h}{vector2.h}

\chapter{Вывод}

Используйте \LaTeX.

\end{document}